% Created 2021-08-13 Fri 15:54
% Intended LaTeX compiler: pdflatex
\documentclass[11pt]{article}
\usepackage[utf8]{inputenc}
\usepackage[T1]{fontenc}
\usepackage{graphicx}
\usepackage{grffile}
\usepackage{longtable}
\usepackage{wrapfig}
\usepackage{rotating}
\usepackage[normalem]{ulem}
\usepackage{amsmath}
\usepackage{textcomp}
\usepackage{amssymb}
\usepackage{capt-of}
\usepackage{hyperref}
\usepackage{jm}
\renewcommand{\HH}{\mathbf{H}}
\newcommand{\DD}{\mathbf{D}}
\newcommand{\MM}{\mathbf{M}}
\newcommand{\PP}{\mathbf{P}}
\newcommand{\mm}{\mathbf{m}}
\newcommand{\oo}{\boldsymbol{\omega}}
\renewcommand{\SS}{\mathbf{S}}
\author{John Mastroberti}
\date{\today}
\title{Qualifying Exam Cram Sheet}
\hypersetup{
 pdfauthor={John Mastroberti},
 pdftitle={Qualifying Exam Cram Sheet},
 pdfkeywords={},
 pdfsubject={},
 pdfcreator={Emacs 27.2 (Org mode 9.5)}, 
 pdflang={English}}
\begin{document}

\maketitle
\tableofcontents

\newpage
\section{Electromagnetism}
\label{sec:orgd2b61a0}
\subsection{Basics}
\label{sec:orgc60413d}
\subsubsection{Maxwell's equations}
\label{sec:org1eb5fcb}
\begin{align*}
\del \cdot \DD & = \rho & \del \cdot \BB & = 0 \\
\del \times \EE & = -\pd{\BB}{t} & \del \times \HH & = \JJ + \pd{\DD}{t}
\end{align*}
\subsubsection{Lorentz Force Law}
\label{sec:org6d87c22}
\begin{align*}
\FF & = q(\EE + \vv \times \BB)
\end{align*}

\subsubsection{Definition of \(\DD\) and \(\HH\)}
\label{sec:org43f57aa}
In terms of \(\PP\) and \(\MM\),
\begin{align*}
\DD & = \eps_0 \EE + \PP \\
\HH & = \frac{1}{\mu_0} \BB - \MM
\end{align*}
In linear media,
\begin{align*}
\DD & = \eps \EE \\
\HH & = \frac{1}{\mu} \BB
\end{align*}
\subsubsection{Potentials}
\label{sec:org8d173ec}
\begin{align*}
\EE & = -\del \Phi &           \Phi(\xxx) & = \frac{1}{4\pi \eps_0} \int \frac{\rho(\xxx')}{\left| \xxx - \xxx' \right|} d^3 x'
                                            + \frac{1}{4\pi \eps_0} \oint \frac{\sigma(\xxx')}{\left| \xxx - \xxx' \right|} da  \\
\BB & = \del \times \AA &      \AA(\xxx) & = \frac{\mu_0}{4\pi} \int \frac{\JJ(\xxx')}{\left| \xxx - \xxx' \right|} d^3 x'
\end{align*}
\subsubsection{Charge Density}
\label{sec:org8b81383}
\begin{align*}
\sigma & = -\eps_0 \pd{\Phi}{\hat{n}}
\end{align*}

\subsection{Method of Images}
\label{sec:orgb6d2f94}
\begin{itemize}
\item Charge \(q\) a distance \(a\) from a grounded conducting plane
\begin{itemize}
\item Image charge \(-q\) a distance \(a\) from the plane
\end{itemize}
\item Charge \(q\) a distance \(y\) from the center of a grounded conducting sphere of radius \(a\)
\begin{itemize}
\item Image charge \(q' = -(a/y) q\) a distance \(a^2/y\) from the center of the sphere
\end{itemize}
\end{itemize}

\subsection{Laplace's Equation and Solutions}
\label{sec:org7893e9f}
In a source free region, both \(\EE\) and \(\HH\) can be written as the gradient of a scalar potential which satisfies Laplace's equation,
\begin{align*}
\nabla^2 \Phi & = 0
\end{align*}
\subsubsection{2D Rectangular Coordinates}
\label{sec:org8afc401}
\begin{align*}
\Phi(x,y) & = \sum_k (A_k \sinh(kx) + B_k \cosh(kx)) (C_k \sin(ky) + D_k \cos(ky))
\end{align*}
Coefficients and allowed values of \(k\) are determined by boundary conditions
\subsubsection{2D Polar Coordinates}
\label{sec:orgf37f899}
\begin{align*}
\Phi(r, \phi) & = \sum_n (A_n r^n + B_n r^{-n}) (C_n \cos n\phi + D_n \sin n\phi)
\end{align*}
\subsubsection{3D Spherical with Azimuthal Symmetry}
\label{sec:org7ce477e}
\begin{align*}
\Phi(r, \theta) & = \sum_l \left( A_l r^l + \frac{B_l}{r^{l+1}} \right) P_l(\cos \theta)
\end{align*}
\subsubsection{3D Spherical without Azimuthal Symmetry}
\label{sec:org5d4bea2}
\begin{align*}
\Phi(r, \theta) & = \sum_{l,m} \left( A_l r^l + \frac{B_l}{r^{l+1}} \right) Y_{lm}(\theta, \phi)
\end{align*}
\subsection{The Addition Theorem}
\label{sec:orgdfa3e4b}
\begin{align*}
\frac{1}{\left| \xxx - \xxx' \right|} & = 4\pi \sum_l \sum_m \frac{1}{2l+1} \frac{r_<^l}{r_>^{l+1}} Y_{lm}^*(\theta', \phi') Y_{lm}(\theta, \phi)
\end{align*}
\subsection{The Multipole Expansion}
\label{sec:org1aef0ee}
\subsubsection{Spherical Coordinates}
\label{sec:org8d848d1}
\begin{align*}
\Phi(\xxx) & = \frac{1}{4\pi \eps_0} \sum_l \sum_m \frac{4\pi}{2l+1} q_{lm} \frac{Y_{lm}(\theta, \phi)}{r^{l+1}} \\
q_{lm} & = \int Y_{lm}^*(\theta', \phi') r^{\prime l} \rho(\xxx') \, d^3 x'
\end{align*}
\subsubsection{Rectangular Dipole and Quadrupole Moments}
\label{sec:org5072087}
\begin{align*}
\pp & = \int \xxx' \rho(\xxx') \, d^3 x' \\
Q_{ij} & = \int (3x_i' x_j' - r^{\prime 2} \delta_{ij}) \rho(\xxx') \, d^3 x' \\
\Phi(\xxx) & = \frac{1}{4\pi \eps_0} \left[ \frac{q}{r} + \frac{\pp \cdot \xxx}{r^3}
+ \frac{1}{2} \sum_{i,j} Q_{ij} \frac{x_i x_j}{r^5} \right]
\end{align*}
\subsection{Magnetism}
\label{sec:orge0d627f}
\subsubsection{Magnetic Field Due to Current Distribution}
\label{sec:org3d769ac}
\begin{align*}
\BB & = \frac{\mu_0}{4\pi} \int \JJ(\xxx') \frac{\xxx - \xxx'}{\left| \xxx - \xxx' \right|} d^3 x'
\end{align*}
\subsubsection{Magnetic Dipole Moment}
\label{sec:orgc98a021}
\begin{align*}
\mm & = \frac{1}{2} \int \xxx' \times \JJ(\xxx') \, d^3 x' \\
m & = I A & \mbox{(for a loop)} \\
\BB & = \frac{\mu_0}{4\pi} \frac{3\nn (\nn \cdot \mm) - \mm}{r^3}
\end{align*}
\subsubsection{Magnetostatic Boundary Value Problems}
\label{sec:org7766e67}
\begin{enumerate}
\item Source Free
\label{sec:org093215d}
\begin{align*}
\HH & = -\del \Phi_M \\
\nabla^2 \Phi_M & = 0
\end{align*}
\item Hard Ferromagnets (\(\MM\) given)
\label{sec:org9d0c2a4}
\begin{align*}
\rho_M & = -\del \cdot \MM \\
\sigma_M & = \nn \cdot \MM \\
\Phi_M(\xxx) & = \frac{1}{4\pi} \int \frac{\rho_M(\xxx')}{\left| \xxx - \xxx' \right|} \, d^3 x'
               + \frac{1}{4\pi} \oint \frac{\sigma_M(\xxx')}{\left| \xxx - \xxx' \right|} \, da
\end{align*}
\end{enumerate}
\subsection{Boundary Value Problem Matching Conditions}
\label{sec:org8d67341}
\begin{align*}
(\DD_2 - \DD_1) \cdot \nn & = \sigma \\
(\EE_2 - \EE_1) \times \nn & = 0 \\
(\BB_2 - \BB_1) \cdot \nn & = 0 \\
(\HH_2 - \HH_1) \times \nn & = -\mathbf{K} \mbox{ (surface current density)}
\end{align*}
\subsection{Field Energy and Momentum}
\label{sec:orga567ae4}
\subsubsection{Energy Density}
\label{sec:orge287a86}
\begin{align*}
u & = \frac{1}{2} (\EE \cdot \DD + \BB \cdot \HH)
\end{align*}
\subsubsection{Momentum Density}
\label{sec:org9cd4bb9}
\begin{align*}
\mathbf{g} & = \frac{1}{c^2} \SS \\
\SS & = \EE \times \HH
\end{align*}
\subsubsection{Energy Conservation}
\label{sec:org171c0a9}
\begin{align*}
\pd{u}{t} + \del \cdot \SS & = -\JJ \cdot \EE
\end{align*}
\subsection{Electromagnetic Waves}
\label{sec:orgb306a65}
\subsubsection{Equations}
\label{sec:orgb1a15c4}
\begin{align*}
\EE & = \EE_0 e^{i(k \nn \cdot \xxx - \omega t)} \\
\HH & = \nn \times \EE / Z
\end{align*}
\subsubsection{Reflection/Refraction}
\label{sec:org50a0975}
\begin{enumerate}
\item Index of Refraction
\label{sec:org3b4a2ff}
\begin{align*}
n & = \sqrt{\frac{\mu \eps}{\mu_0 \eps_0}}
\end{align*}
\item Normal Incidence
\label{sec:org8d2b40c}
\begin{itemize}
\item Transverse components of \(\EE\) and \(\HH\) are continuous
\item Reflection and transmission coefficients:
\end{itemize}
\begin{align*}
T & = \frac{2n}{n' + n} \\
R & = \pm \frac{n-n'}{n+n'} \\
\end{align*}
\end{enumerate}
\subsection{Radiation}
\label{sec:org6849536}
\subsubsection{Power Radiated}
\label{sec:org6dc919e}
\begin{align*}
\td{P}{\Omega} & = \frac{1}{2} \mathrm{Re} [r^2 \nn \cdot \EE \times \HH^*] \\
& = \frac{c^2 Z_0}{32 \pi^2} k^4 \left| (\nn \times \pp) \times \nn \right|^2 \mbox{ (for dipole radiation)}
\end{align*}
\subsection{Relativity}
\label{sec:orgb92af09}
\begin{align*}
\gamma & = \frac{1}{\sqrt{1 - \beta^2}} \\
\Gamma & = \left( \begin{array}{cccc}
                  \gamma & \pm \gamma \beta & 0 & 0 \\
                  \pm \gamma \beta & \gamma & 0 & 0 \\
                  0 & 0 & 1 & 0 \\
                  0 & 0 & 0 & 1 \\ \end{array} \right) \\
p_\mu p^\mu & = m^2 \\
E & = \gamma m c^2 \\
\pp & = \gamma m \uu \\
U_\mu & = (\gamma c, \gamma \uu) \\
p_\mu & = m U_\mu
\end{align*}





\newpage
\section{Mechanics}
\label{sec:org1deb8d6}
\subsection{Basics}
\label{sec:org2c2db82}
\begin{align*}
\FF & = \dot{\pp} \\
\boldsymbol{\Gamma} & = \rrr \times \FF \\
\LL & = \rrr \times \pp
\end{align*}
\subsection{Orbital Motion}
\label{sec:org11ec163}
\begin{align*}
V_{eff}(r) & = V(r) + \frac{\ell^2}{2mr^2}
\end{align*}
\begin{itemize}
\item Circular orbits \(\rightarrow\) minimum of \(V_{eff}\)
\item Minimum and maximum \(r\) values for a non-circular orbit \(\rightarrow\) solutions of \(V_{eff} = E\)
\end{itemize}
\begin{align*}
\phi & = \pm \frac{\ell}{\sqrt{2m}} \int dr \, r^{-2} [E - V_{eff}(r)]^{-1/2}
\end{align*}
\subsection{Non-inertial Coordinate Systems}
\label{sec:org2b5bf5c}
\begin{align*}
\ddot{\rrr}_{body} & = \frac{\FF^{(e)}}{m} - \ddot{\mathbf{a}}_{inertial} - 2\oo \times \dot{\rrr}_{body}
- \oo \times (\oo \times \rr) - \dot{\oo} \times \rrr
\end{align*}
\subsection{Lagrangian Dynamics}
\label{sec:org0b406aa}
\begin{align*}
\Lg & = T - V \\
\pd{\Lg}{q} & = \td{}{t} \pd{\Lg}{\dot{q}}
\end{align*}

With constraints
\begin{align*}
f_j(q_1, \ldots, q_n, t) & = c_j, \, j = 1, \ldots, k
\end{align*}
the Euler-Lagrange equation becomes
\begin{align*}
\td{\Lg}{\dot{q}_\sigma} - \pd{\Lg}{q_\sigma} & = \sum_{j=1}^k \lambda_j \pd{f_j}{q_\sigma}, \, \sigma = 1, \ldots, n
\end{align*}
where the constraint forces are given by the right hand side,
\begin{align*}
Q_\sigma & = \sum_{j=1}^k \lambda_j \pd{f_j}{q_\sigma}
\end{align*}

\subsection{Small Oscillations}
\label{sec:org81e93e2}
First, expand the coordinates around their equilibrium values:
\begin{align*}
q_\sigma & = q_\sigma^0 + \eta_\sigma \\
\dot{q}_\sigma & = \dot{\eta}_\sigma
\end{align*}
Then, working to quadratic order in \(\eta\), write \(T\) and \(V\) as
\begin{align*}
T & = \frac{1}{2} \sum_\sigma \sum_\lambda m_{\sigma \lambda} \dot{\eta}_\sigma \dot{\eta}_\lambda \\
V & = \frac{1}{2} \sum_\sigma \sum_\lambda v_{\sigma \lambda} \eta_\sigma \eta_\lambda
\end{align*}
This determines the \(\underline{m}\) and \(\underline{v}\) matrices.
Finally, the normal mode frequencies are found via
\begin{align*}
0 & = \det [ \underline{v} - \omega^2 \underline{m} ]
\end{align*}
and the normal mode eigenvectors \(\boldsymbol{\rho}\) via
\begin{align*}
0 & = (\underline{v} - \omega^2 \underline{m}) \boldsymbol{\rho}
\end{align*}

\subsection{Rigid Body Rotations}
\label{sec:org0b66435}
\begin{align*}
\LL & = \underline{I} \oo \\
T & = \frac{1}{2} \LL \cdot \oo \\
I_1 \dot{\omega}_1 & = \omega_2 \omega_3 (I_2 - I_3) + \Gamma_1^{(e)} \mbox{ (and cyclic permutations)}
\end{align*}


\newpage
\section{Quantum Mechanics}
\label{sec:org924cdb3}
\subsection{Basics}
\label{sec:org00c0265}
\begin{align*}
i\hbar \pd{}{t} \Psi(\xxx,t) & = H\Psi(\xxx, t) \\
H\psi(\xxx) & = E\psi(\xxx) \\
\psi(\xxx, t) & = \sum_n c_n \psi_n(\xxx) e^{-iEt/\hbar} \\
H & = \frac{p^2}{2m} + V \\
p & = -i\hbar \pd{}{x} \\
[x, p] & = i\hbar \\
\sigma_A \sigma_B & \ge \left| \frac{1}{2i} \Braket{[A, B]} \right|
\end{align*}
\begin{itemize}
\item Matching conditions: \(\psi(x)\) and \(\psi'(x)\) are continuous
\begin{itemize}
\item \(\psi'(x)\) is discontinous in the presence of a delta function potential
\item In that case, determine the discontinuity in \(\psi'(x)\) by integrating the Schrodinger
equation over a small interval surrounding the delta function
\end{itemize}
\end{itemize}

\subsection{Common Potentials}
\label{sec:orgb9f889a}
\subsubsection{Particle in a Box}
\label{sec:orgd12316d}
\begin{align*}
V(x) & = \begin{dcases*}
            0 & $x \in [0, a]$ \\
            \infty & otherwise \\
         \end{dcases*} \\
\psi_n(x) & = \sqrt{\frac{2}{a}} \sin \frac{n \pi x}{a} \\
E_n & = \frac{\hbar^2 n^2 \pi^2}{2ma^2}
\end{align*}
\subsubsection{Simple Harmonic Oscillator}
\label{sec:orgd77ad02}
\begin{align*}
V(x) & = \frac{1}{2} m \omega^2 x^2 \\
a & = \frac{1}{\sqrt{2\hbar m \omega}} (m\omega x + ip) \\
[a, a^\dagger] & = 1 \\
N & = a^\dagger a \\
H & = \hbar \omega \left( N + \frac{1}{2} \right) \\
E_n & = \hbar \omega \left( n + \frac{1}{2} \right) \\
\psi_0(x) & = \left( \frac{m\omega}{\pi \hbar} \right)^{1/4} \exp \left( - \frac{m\omega}{2\hbar} x^2 \right) \\
\psi_n(x) & = \frac{1}{\sqrt{n!}} (a^\dagger)^n \psi_0(x) \\
a^\dagger \Ket{n} & = \sqrt{n+1} \Ket{n+1} \\
a \Ket{n} & = \sqrt{n} \Ket{n-1} \\
x & \propto a + a^\dagger \\
p & \propto a - a^\dagger
\end{align*}
\subsubsection{Hydrogen Atom}
\label{sec:orgf45ab2e}
\begin{align*}
V(\xxx) & = -\frac{Ze^2}{r} \\
\psi_{nlm}(\xxx) & = R_{nl}(r) Y_{lm}(\theta, \phi) \\
E_n & = -\frac{Z^2 e^2}{2n^2 a_0} \\
E_1 & = -13.6 \mathrm{eV}
\end{align*}

\subsection{Angular Momentum}
\label{sec:orgb7a60a3}
\subsubsection{Basics}
\label{sec:org15bc129}
\begin{align*}
[L_x, L_y] & = i\hbar L_z \\
L_z \Ket{l, m} & = \hbar m \Ket{l, m} \\
L^2 \Ket{l, m} & = \hbar^2 l (l+1) \Ket{l, m} \\
L_{\pm} & = L_x \pm i L_y \\
L_{\pm} \Ket{l, m} & = \hbar \sqrt{(l\mp m) (l \pm m + 1)} \Ket{l, m \pm 1} \\
\sigma_x & = \left( \begin{array}{cc} 0 & 1 \\ 1 & 0 \\ \end{array} \right) \\
\sigma_y & = \left( \begin{array}{cc} 0 & -i \\ i & 0 \\ \end{array} \right) \\
\sigma_z & = \left( \begin{array}{cc} 1 & 0 \\ 0 & -1 \\ \end{array} \right)
\end{align*}

\begin{itemize}
\item Allowed values of \(m\): \(-l, -l+1, \ldots, +l\)
\end{itemize}

\subsubsection{Addition of Angular Momentum}
\label{sec:org3df59ec}
\begin{align*}
\JJ & = \LL_1 + \LL_2 \\
J^2 & = L_1^2 + L_2^2 + 2 \LL_1 \cdot \LL_2 \\
\LL_1 \cdot \LL_2 & = \frac{1}{2} ( J^2 - L_1^2 - L_2^2 ) \\
\Ket{j, j_z} & = \sum C_{l_1, l_2, m_1, m_2} \Ket{l_1, m_1; l_2, m_2} \\
j_z & = m_1 + m_2 \\
j & = l_1 + l_2, l_1 + l_2 -1, \ldots, \left| l_1 - l_2 \right| \\
\Ket{l_1 + l_2, l_1 + l_2} & = \Ket{l_1, l_1; l_2, l_2}
\end{align*}

\begin{itemize}
\item \(\Ket{l_1+l_2 , j_z}\) is constructed by repeatedly applying \(J_-\) to \(\Ket{j, j}\).
\item \(\Ket{j, j_z}\) is constructed by walking down the allowed values of \(j\) starting from \(l_1 + l_2\)
\begin{itemize}
\item Construct \(\Ket{j-1, j-1}\) from \(\Ket{j, j-1}\) by finding the linear combination of allowed \(m_1\) and \(m_2\)
that is orthogonal to \(\Ket{j, j-1}\)
\item Construct \(\Ket{j-1, j_z}\) by applying the lowering operator
\end{itemize}
\end{itemize}

\subsection{Approximation Methods}
\label{sec:org14ab909}
\subsubsection{Perturbation Theory}
\label{sec:org109b750}
\begin{align*}
H & = H_0 + V \\
V_{mn} & = \Braket{m^{(0)} | V | n^{(0)}} \\
E_n^{(1)} & = V_{nn} \\
\Ket{n^{(1)}} & = \sum_{m \neq n} \frac{V_{mn}}{E_n^{(0)} - E_m^{(0)}} \Ket{m^{(0)}} \\
E_n^{(2)} & = \sum_{m \neq n} \frac{\left| V_{nm} \right|^2}{E_n^{(0)} - E_m^{(0)}}
\end{align*}

\begin{itemize}
\item In a degenerate subspace where all states have energy \(E_D\) under \(H_0\), the first order correction to the energy is found via
\end{itemize}

\begin{align*}
0 & = \det [V - (E-E_D)]
\end{align*}

\subsubsection{Variational Principle}
\label{sec:orga1bbaa2}
\begin{align*}
E_0 & \le \frac{\Braket{\psi | H | \psi}}{\Braket{\psi | \psi}}
\end{align*}

\subsubsection{WKB}
\label{sec:orge2304e7}
\begin{align*}
p(x) & = \sqrt{2m(E - V(x)} \\
\psi(x) & \sim \frac{C}{\sqrt{p(x)}} e^{\pm \frac{i}{\hbar} \int p(x) \, dx} \\
\int_{x_1}^{x_2} p(x') \, dx' & = \begin{dcases*}
                                        \left( n + \frac{1}{2} \right) \hbar \pi & no hard walls \\
                                        \left( n + \frac{3}{4} \right) \hbar \pi & one hard wall \\
                                        \left( n + 1 \right)           \hbar \pi & two hard walls \\
                                  \end{dcases*}
\end{align*}
where \(x_1\) and \(x_2\) are the classical turning points (roots of \(p(x)\))

\subsection{Scattering (Born Approximation)}
\label{sec:orga7d2fae}
\begin{align*}
f^{(1)}(\mathbf{k}, \mathbf{k}') & = -\frac{m}{2\pi \hbar^2} \int d^3 x' \, e^{i(\mathbf{k} - \mathbf{k}') \cdot \xxx'} V(\xxx') \\
q & = \left| \mathbf{k} - \mathbf{k}' \right| = 2k \sin \frac{\theta}{2} \\
f^{(1)}(\theta) & = -\frac{2m}{\hbar^2} \frac{1}{q} \intoi r V(r) \sin qr \, dr \\
\td{\sigma}{\Omega} & = \left| f(\mathbf{k}', \mathbf{k}) \right|^2 \\
\mathrm{Im} f(\mathbf{k}, \mathbf{k}) & = \frac{k}{4\pi} \sigma_{tot}
\end{align*}

\subsection{Many Body Theory}
\label{sec:org1a952ed}
\begin{itemize}
\item Boson wave functions are symmetric under any particle exchange
\item Fermion wave functions are antisymmetric under any particle exchange
\item For wave-functions with a spatial component and a spin component,
\begin{itemize}
\item Boson: symmetric in both or antisymmetric in both
\item Fermion: symmetric in one and antisymmetric in the other
\end{itemize}
\end{itemize}


\newpage
\section{Statistical Physics}
\label{sec:org881e47d}
\subsection{Basics}
\label{sec:org001d880}
\begin{align*}
S & = k_B \log \Omega \\
\pdc{S}{E}{N,V} & = \frac{1}{T} \\
dE & = -PdV + TdS + \mu dN \\
C_V & = \pdc{U}{T}{V, N}
\end{align*}
\begin{itemize}
\item Isothermal: \(dT = 0\)
\item Adiabatic: \(dS = 0\)
\end{itemize}

\subsection{Canonical Ensemble}
\label{sec:org0caeda0}
\begin{align*}
Z & = \sum_{states} e^{-\beta E} \\
\Braket{O} & = \frac{1}{Z} \sum_s O(s) e^{-\beta E} \\
\sum_s & \rightarrow \int \frac{d^3 x \, d^3 p}{h^3} \\
U & = -\pd{}{\beta} \log Z \\
A & = U - TS = -k_B T \log Z \\
dA & = -P dV - S dT
\end{align*}

\begin{itemize}
\item Partition function for an ideal gas molecule:
\begin{align*}
  Z_1 & = \frac{V}{\lambda^3} \\
  \lambda & = \sqrt{\frac{h^2}{2\pi m k_B T}}
\end{align*}
\item In general, for a system of non-interacting particles, the total partition function obeys
\begin{align*}
  Z & = \frac{1}{N!} Z_1^N
\end{align*}
where \(Z_1\) is the partition function for one particle,
and the \(1/N!\) factor is necessary if the particles are indistinguishable
\end{itemize}

\subsection{Grand Canonical Ensemble}
\label{sec:org65e336c}
\begin{align*}
Z_G & = \sum_{states} e^{-\alpha N - \beta E} \\
U & = -\pd{}{\beta} \log Z_G \\
\mathcal{N} & = -\pd{}{\alpha} \log Z_G \\
\alpha & = -\beta \mu \\
y & = e^{-\alpha} = e^{\mu/k_B T} \\
Z_G & = \sum_N y^N Z(N, \beta) \\
\Phi_G & = -k_B T \log Z_G = U - TS - \mu \mathcal{N} = -PV
\end{align*}

\subsection{Quantum Statistical Mechanics}
\label{sec:orgd2cb573}
\begin{align*}
Z_G & = \prod_n [1 - \eta e^{-\beta (E_n - \mu)} ]^{-\eta} \\
\Braket{N_n} & = \frac{1}{y^{-1} e^{\beta E_n} - \eta} \\
D(E) & = \frac{g(E)}{V} = \frac{1}{V} \sum_\mathbf{k} (E - E_\mathbf{k}) \\
\varrho & = \int \frac{D(E)}{y^{-1} e^{\beta E} - 1} \, dE \\
\log Z_G & = -\eta V \intoi \log ( 1 - \eta ye^{-\beta E} ) D(E) \, dE
\end{align*}
\begin{itemize}
\item \(\eta = +1\) for bosons, \(-1\) for fermions
\end{itemize}

\subsubsection{Bose Systems}
\label{sec:org5360079}
\begin{itemize}
\item \(y=1\) at Bose-Einstein Condensation, and in all cases for photons
\end{itemize}

\subsubsection{Fermi Systems}
\label{sec:org1eadd4a}
\begin{align*}
\varrho & = \int_0^{E_F} D(E) \, dE
\end{align*}
\begin{itemize}
\item Any other Fermi parameter (Fermi temperature, Fermi momentum, etc)
is determined from the relationship between energy and that parameter
(E.g. \(T_F = E_F/k_B\))
\end{itemize}
\end{document}
