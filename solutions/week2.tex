\documentclass[12pt]{article}

\usepackage{jm}

% This changes the first-level of an enumerated list to use letters instead of numbers
\renewcommand{\theenumi}{\alph{enumi}}
\begin{document}

\title{Week 2}
\author{John Mastroberti\\
Qual Study}

\maketitle

\begin{solution}{2018.11}
The ground state wave function should be symmetric about $x=0$.
In the regions of interest ($\left| x \right| > a$ and $0 < \left| x \right| < a$,
Schr\"{o}dinger's equation reads
\begin{align*}
\begin{dcases*}
-\frac{\hbar^2}{2m} \psi'' = E \psi & $\left| x \right| > a$ \\
-\frac{\hbar^2}{2m} \psi'' + V_1 \psi = E \psi & $0 < \left| x \right| < a$ \\
\end{dcases*}
\end{align*}
Defining $k_1$ and $k_2$ by
\begin{align*}
k_1 & = \sqrt{ -\frac{2mE}{\hbar^2}} \\
k_1 & = \sqrt{ -\frac{2m(E-V_1)}{\hbar^2}} \\
\end{align*}
we see that the solutions in the two regions are
\begin{align*}
\psi(x) & = \begin{dcases*}
A e^{-k_1 \left| x \right|} & $\left| x \right| > a$ \\
B e^{-k_2 \left| x \right|} + C e^{k_2 \left| x \right|} & $0 < \left| x \right| < a$ \\
\end{dcases*}
\end{align*}
Applying matching conditions at $\left| x \right| = a$ gives us
\begin{align*}
A e^{-k_1 a} & = B e^{-k_2 a} + C e^{k_2 a} \\
% C & = A e^{-k_1 a -k_2 a} - B e^{-2k_2 a} \\
-k_1 A e^{-k_1 a} & = -k_2 B e^{-k_2 a} + k_2 C e^{k_2 a} \\
0 & = \frac{k_2}{k_1} e^{-k_2 a} B - \frac{k_2}{k_1} e^{k_2 a} C \\
B & = e^{2k_2 a} C
\end{align*}
Due to the delta function at $x=0$ in the potential, $\psi'$ is not continuous at $x=0$.
In the region $0 < \left| x \right| < a$, we have
\begin{align*}
\psi'(x) & = \left(-B k_2 e^{-k_2 \left| x \right|} 
+ C k_2 e^{k_2 \left| x \right|} \right) \frac{\left| x \right|}{x} \\
\Delta \psi' & \equiv \lim_{\eps \rightarrow 0} (\psi'(\eps) - \psi'(-\eps)) 
= 2 k_2 (C - B)
\end{align*}
We also see from integrating the Schr\"{o}dinger equation over the region $(-\eps, \eps)$ that
\begin{align*}
-\frac{\hbar^2}{2m} \Delta \psi' + V_0 \psi(0) & = 0
\end{align*}
Thus,
\begin{align*}
2k_2 (C - B) & = \frac{2m}{\hbar^2} V_0 \psi(0) \\
& = \frac{2m}{\hbar^2} V_0 (B+C) \\
2k_2 (1 - e^{2k_2 a}) & = \frac{2m}{\hbar^2} V_0 (e^{2k_2 a} + 1)
\end{align*}
This equation can be solved for $k_2$, which, using the definition of $k_2$, yields the ground state energy $E$.
\end{solution}




\begin{solution}{2018.13}
The wave function in each region is
\begin{align*}
\psi(x) & = \begin{dcases*}
A e^{-ikx} & $x < 0$ \\
B \cos (k' x) + C \sin (k' x) & $0 < x < a$ \\
D e^{ik x} + E e^{-ik x} & $x > a$
\end{dcases*} \\
k & = \frac{\sqrt{2mE}}{\hbar} \\
k' & = \frac{\sqrt{2m(E-V_0)}}{\hbar}
\end{align*}
$E e^{-ikx}$ corresponds to the incident wave, so the transmission coefficient is
$\left| A \right|^2 / \left| E \right|^2$.
Applying matching conditions at $x=0$ and $x=a$, we find that
\begin{align*}
A & = B + 0 \\
-ikA & = k' C \\
B \cos (k' a) + C \sin (k'a) & = D e^{ika} + E e^{-ika} \\
-k'B \sin (k' a) + k' C \cos (k'a) & = ik D e^{ika} - ikE e^{-ika} \\
\end{align*}
Clearly, $B = A$ and $C = -i(k/k') A$. Then
\begin{align*}
D e^{ika} - E e^{-ika} 
& = \frac{1}{ik} \left[ -k'B \sin (k' a) + k' C \cos (k'a) \right] \\
2E e^{-ika} & = B \cos (k'a) + C \sin (k' a) 
- \frac{1}{ik} \left[ -k'B \sin (k' a) + k' C \cos (k'a) \right] \\
& = A \cos (k'a) - i \frac{k}{k'} A \sin (k' a) 
- \frac{1}{ik} \left[ -k'A \sin (k' a) - i \frac{k}{k'} k' A \cos (k'a) \right] \\
& = A \cos (k'a) - i \frac{k}{k'} A \sin (k' a) 
-i \frac{k'}{k} A \sin (k' a) + A \cos (k'a) \\
\frac{E}{A} & = e^{ika} \left[ \cos (k'a) - i \frac{k}{k'} \sin (k' a) \right] \\
\frac{\left| E \right|^2}{\left| A \right|^2} & = \cos^2 (k'a) + \left( \frac{k}{k'} \right)^2 \sin^2 (k'a)
\end{align*}
Therefore, the transmission coefficient is 100\% whenever
\begin{align*}
1 & = \cos^2 (k'a) + \left( \frac{k}{k'} \right)^2 \sin^2 (k'a) \\
\sin^2 (k' a) & = \left( \frac{k}{k'} \right)^2 \sin^2 (k'a) \\
k & = k'
\end{align*}
Probably an algebra mistake somewhere.
\end{solution}



\begin{solution}{2017.12}
\begin{enumerate}
\item
It should be even under parity

\item
I don't wanna

\end{enumerate}

\end{solution}




\begin{solution}{2017.14}
Since
\begin{align*}
\Ket{\psi(t=0)} & = \frac{1}{\sqrt{2}} \Ket{u_1} + \frac{1}{\sqrt{2}} \Ket{u_2}
\end{align*}
we have
\begin{align*}
\Ket{\psi(t)} & = \frac{1}{\sqrt{2}} e^{-iE_1 t/\hbar} \Ket{u_1} 
+ \frac{1}{\sqrt{2}} e^{-iE_2 t / \hbar} \Ket{u_2}
\end{align*}
Finally, we write $A$ in the energy eigenbasis,
\begin{align*}
\hat{A} & = x \left( \begin{array}{cc}
1 & 0 \\
0 & 1 \\
\end{array} \right) + y \left( \begin{array}{cc}
0 & 1 \\
1 & 0 \\
\end{array} \right) \\
\hat{A} \Ket{\phi_1} & = x \Ket{\phi_1} + y \Ket{\phi_1} = a_1 \Ket{\phi_1} \\
\hat{A} \Ket{\phi_2} & = x \Ket{\phi_2} - y \Ket{\phi_2} = a_2 \Ket{\phi_2} \\
x & = \frac{a_1 + a_2}{2} \\
y & = \frac{a_1 - a_2}{2} \\
\hat{A} & = \frac{1}{2} \left( \begin{array}{cc}
a_1 + a_2 & a_1 - a_2 \\
a_1 - a_2 & a_1 + a_2 \\
\end{array} \right)
\end{align*}
so that we have
\begin{align*}
\Braket{A(t)} & = \Braket{\psi(t) | \hat{A} | \psi(t)} \\
& = \frac{1}{4} \left( \begin{array}{cc}
e^{iE_1 t/\hbar} & e^{i E_2 t/\hbar} \\
\end{array} \right) \left( \begin{array}{cc}
a_1 + a_2 & a_1 - a_2 \\
a_1 - a_2 & a_1 + a_2 \\
\end{array} \right) \left( \begin{array}{c}
e^{-iE_1 t/\hbar} \\
e^{-iE_2 t/\hbar}
\end{array} \right) \\
& = \frac{1}{4} \left( \begin{array}{cc}
e^{iE_1 t/\hbar} & e^{i E_2 t/\hbar} \\
\end{array} \right) \left( \begin{array}{c}
(a_1 + a_2) e^{-iE_1 t/\hbar} + (a_1 - a_2) e^{-iE_2 t/\hbar} \\
(a_1 - a_2) e^{-iE_1 t/\hbar} + (a_1 + a_2) e^{-iE_2 t/\hbar} \\
\end{array} \right) \\
& = \frac{1}{4} \left[ (a_1 + a_2) + (a_1 - a_2) e^{i(E_1 - E_2) t/\hbar}
+ (a_1 - a_2) e^{-i(E_1 - E_2) t/\hbar} + (a_1 + a_2) \right] \\
& = \frac{1}{2} \left[ (a_1 + a_2) + (a_1 - a_2) \cos \frac{(E_1 - E_2) t}{\hbar} \right] \\
& = a_1 \cos \frac{(E_1 - E_2)t}{2\hbar} + a_2 \sin \frac{(E_1 - E_2)t}{2\hbar}
\end{align*}


\end{solution}



\begin{solution}{2016.11}
\begin{enumerate}
\item
Schr\"{o}dinger's equation reads
\begin{align*}
-\frac{\hbar^2}{2m} \psi'' - \frac{e^2}{4x} \psi & = E \psi
\end{align*}
Taking the hint's suggestion, we let $\psi(x) = u(x) e^{-kx}$ for $x>0$ 
($\psi(x) = 0$ for $x<0$), so that
\begin{align*}
\psi'(x) & = u'(x) e^{-kx} - ku(x) e^{-kx} \\
\psi''(x) & = u''(x) e^{-kx} - 2ku'(x) e^{-kx} + k^2 u(x) e^{-kx} \\
Eu & = -\frac{\hbar^2}{2m} (u'' - 2ku' + k^2 u) - \frac{e^2}{4x} u
\end{align*}


\end{enumerate}

\end{solution}




\begin{solution}{2016.13}
\begin{enumerate}
\item
First, we write the energy eigenstates in terms of the flavor eigenstates:
\begin{align*}
\left( \begin{array}{c}
\Ket{\nu_e} \\
\Ket{\nu_\mu} \\
\end{array} \right) & = \left( \begin{array}{cc}
\cos \theta & \sin \theta \\
-\sin \theta & \cos \theta \\
\end{array} \right) \left( \begin{array}{c}
\Ket{\nu_1} \\
\Ket{\nu_2} \\
\end{array} \right) \\
\left( \begin{array}{c}
\Ket{\nu_1} \\
\Ket{\nu_2} \\
\end{array} \right) & = \left( \begin{array}{cc}
\cos \theta & -\sin \theta \\
\sin \theta & \cos \theta \\
\end{array} \right) \left( \begin{array}{c}
\Ket{\nu_e} \\
\Ket{\nu_\mu} \\
\end{array} \right)
\end{align*}
Then, since
\begin{align*}
H & = E_1 \Ket{\nu_1} \Bra{\nu_1} + E_2 \Ket{\nu_2} \Bra{\nu_2}
\end{align*}
we can write
\begin{align*}
H & = E_1 \left( \cos \theta \Ket{\nu_e} - \sin \theta \Ket{\nu_\mu} \right)
\left( \cos \theta \Bra{\nu_e} - \sin \theta \Bra{\nu_\mu} \right) \\
& + E_2 \left( \sin \theta \Ket{\nu_e} + \cos \theta \Ket{\nu_\mu} \right)
\left( \sin \theta \Bra{\nu_e} + \cos \theta \Bra{\nu_\mu} \right) \\
& = (E_1 + E_2) \Ket{\nu_e} \Bra{\nu_e} + (E_2 + E_1) \Ket{\nu_\mu} \Bra{\nu_\mu} \\
& + (E_2 - E_1) \sin \theta \cos \theta \Ket{\nu_\mu} \Bra{\nu_e}
+ (E_2 - E_1) \sin \theta \cos \theta \Ket{\nu_e} \Bra{\nu_\mu}
\end{align*}
or, as a matrix in the flavor basis $\{\Ket{\nu_e}, \Ket{\nu_\mu}\}$,
\begin{align*}
H & = \left( \begin{array}{cc}
E_1 + E_2 & (E_2 - E_1) \sin \theta \cos \theta \\
(E_2 - E_1) \sin \theta \cos \theta & E_1 + E_2 \\
\end{array} \right)
\end{align*}
Defining $a = E_1 + E_2$ and $b = (E_2 - E_1) \sin \theta \cos \theta$ for convenience, we then have that the time evolution operator in the flavor basis is
\begin{align*}
U(t, t_0 = 0) & = e^{-iH t/\hbar} \\
& = \sum_{n=0}^\infty \frac{1}{n!} \left( -\frac{it}{\hbar} \right)^n \left( \begin{array}{cc}
a & b \\
b & a \\
\end{array} \right)^n \\
\left( \begin{array}{cc}
a & b \\
b & a \\
\end{array} \right)^2 & = \left( \begin{array}{cc}
a^2 + b^2 & 2ab \\
2ab & a^2 + b^2
\end{array} \right) \\
2ab & = (E_2^2 - E_1^2) \sin 2\theta \\
a^2 + b^2 & = (E_1 + E_2)^2 + \frac{1}{4} (E_2 - E_1)^2 \sin^2 2\theta
\end{align*}


\end{enumerate}

\end{solution}



\begin{solution}{2015.12}
The ground state of $H_0$ is
\begin{align*}
\psi(x) & = \left( \frac{m\omega}{\pi \hbar} \right)^{1/4} e^{-m\omega x^2/2\hbar}
\end{align*}
In free space, Schr\"{o}dinger's equation is
\begin{align*}
i\hbar \pd{}{t} \Psi(x, t) & = -\frac{\hbar^2}{2m} \pd{^2}{x^2} \Psi(x,t)
\end{align*}
Taking the Fourier transform of this equation gives us
\begin{align*}
i\hbar \pd{}{t} \frac{1}{\sqrt{2\pi}} \intii \Psi(x, t) e^{-ikx} \, dx
& = -\frac{\hbar^2}{2m} \frac{1}{\sqrt{2\pi}} \intii \left( \pd{^2}{x^2} \Psi(x, t) \right) e^{-ikx} \, dx \\
i\hbar \pd{}{t} \Phi(k, t) & = \frac{\hbar^2 k^2}{2m} \Phi(k, t) \\
\Phi(k, t) & = \Phi(k, 0) e^{-i\frac{\hbar k^2}{2m} t} \\
\Phi(k, 0) & = \frac{1}{\sqrt{2\pi}} \intii \left( \frac{m\omega}{\pi \hbar} \right)^{1/4} e^{-m\omega x^2/2\hbar} e^{-ikx} \, dx
\end{align*}
To finish, you would do the integral for $\Phi(k, 0)$, then take the inverse Fourier transform of $\Phi(k, t)$ to get $\Psi(x, t)$.
$\left| \Psi(x, t) \right|^2 dx$ would then be the desired probability.
\end{solution}




\begin{solution}{2015.14}
\end{solution}



\renewcommand{\kk}{\mathbf{k}}
\begin{solution}{2014.11}
\begin{enumerate}
\item
In this case,
\begin{align*}
f^{(1)} (\kk', \kk) & = -\frac{1}{4\pi} \frac{2m}{\hbar^2} \int d^3 x' \, e^{i(\kk - \kk') \cdot \xxx'} V(\xxx') \\
& = -\frac{1}{4\pi} \frac{2m}{\hbar^2} \left[ A - \frac{\hbar^2}{4} B \right] \\
\td{\sigma}{\Omega} & = \left[ \frac{1}{4\pi} \frac{2m}{\hbar^2} \left( A - \frac{\hbar^2}{4} B \right) \right]^2
\end{align*}

\item
In this case,
\begin{align*}
f^{(1)} (\kk', \kk) & = -\frac{1}{4\pi} \frac{2m}{\hbar^2} \int d^3 x' \, e^{i(\kk - \kk') \cdot \xxx'} V(\xxx') \\
& = -\frac{1}{4\pi} \frac{2m}{\hbar^2} \left[ A + \frac{\hbar^2}{4} B \right] \\
\td{\sigma}{\Omega} & = \left[ \frac{1}{4\pi} \frac{2m}{\hbar^2} \left( A + \frac{\hbar^2}{4} B \right) \right]^2
\end{align*}

\item


\end{enumerate}

\end{solution}




\begin{solution}{2014.13}
\begin{enumerate}
\item
Using the ordering $\{ \Ket{00}, \Ket{01}, \Ket{10}, \Ket{11} \}$, the cNOT gate is written as
\begin{align*}
U_{cNOT} & = \left( \begin{array}{cccc}
1 & 0 & 0 & 0 \\
0 & 1 & 0 & 0 \\
0 & 0 & 0 & 1 \\
0 & 0 & 1 & 0 \\
\end{array} \right)
\end{align*}

\item
The state after this operation is
\begin{align*}
\Ket{\psi_f} & = U_{cNOT} U_H \Ket{\psi_i} \\
& = \left( \begin{array}{cccc}
1 & 0 & 0 & 0 \\
0 & 1 & 0 & 0 \\
0 & 0 & 0 & 1 \\
0 & 0 & 1 & 0 \\
\end{array} \right) \left( \begin{array}{c}
1/\sqrt{2} \\
0 \\
1/\sqrt{2} \\
0 \\
\end{array} \right) \\
& = \left( \begin{array}{c}
1/\sqrt{2} \\
0 \\
0 \\
1/\sqrt{2} \\
\end{array} \right) \\
& = \frac{1}{\sqrt{2}} \left( \Ket{00} + \Ket{11} \right)
\end{align*}

\item
You are equally likely to measure 0 or 1 for the first bit.
However, after making a measurement of the first bit, the second bit must be in the same state as the first bit.

\end{enumerate}


\end{solution}



\begin{solution}{2013.12}

\end{solution}




\begin{solution}{2013.14}
\begin{itemize}
\item
$N \eps_1$ \\

\item
$2\eps_1 + 2\eps_2 + \cdots + 2\eps_{N/2}$

\item
(a) $\Ket{\phi_1} \otimes \Ket{\phi_1} \otimes \Ket{\phi_1}$

(b) Two fold degeneracy
\begin{align*}
\frac{1}{\sqrt{6}} \left( 
\Ket{1\uparrow ; 1\downarrow ; 2\uparrow} 
+\Ket{1\downarrow ; 2\uparrow ; 1\uparrow} 
+\Ket{2\uparrow ; 1\uparrow ; 1\downarrow} 
-\Ket{1\downarrow ; 1\uparrow ; 2\uparrow} 
-\Ket{2\uparrow ; 1\downarrow ; 1\uparrow} 
-\Ket{1\uparrow ; 2\uparrow ; 1\downarrow} 
\right)
\end{align*}


\end{itemize}

\end{solution}



\end{document}
