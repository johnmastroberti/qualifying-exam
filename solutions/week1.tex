\documentclass[12pt]{article}

\usepackage{jm}

% This changes the first-level of an enumerated list to use letters instead of numbers
\renewcommand{\theenumi}{\alph{enumi}}
\begin{document}

\title{Week 1}
\author{John Mastroberti\\
Qual Study}

\maketitle

\begin{solution}{2018.1}
Since we need the constraint forces for part (c), we may as well work the problem using Lagrange multipliers from the start.
In each case, the unconstrained Lagrangian is
\begin{align*}
\Lg & = \frac{1}{2} m (\dot{x}^2 + \dot{y}^2) - mgy
\end{align*}

\begin{enumerate}
\item
The constraint is $f(x,y) = y - ax^2 = 0$, so we have
\begin{align*}
\td{}{t} \pd{\Lg}{\dot{x}} - \pd{\Lg}{x} & = \lambda \pd{f}{x} \\
m \ddot{x} & = -2\lambda a x \\
\td{}{t} \pd{\Lg}{\dot{y}} - \pd{\Lg}{y} & = \lambda \pd{f}{y} \\
m \ddot{y} + mg & = \lambda \\
\end{align*}
Let $\lambda' = \lambda/m$ to clean things up.
Our system of equations is then
\begin{align*}
\ddot{x} & = -2\lambda' ax \\
\ddot{y} & = -g + \lambda' \\
y & = ax^2 \\
\end{align*}
The constraint tells us that
\begin{align*}
\dot{y} & = 2ax \dot{x} \\
\ddot{y} & = 2a (\dot{x}^2 + x \ddot{x})
\end{align*}
so we have
\begin{align*}
2a (\dot{x}^2 + x \ddot{x}) & = g + \lambda' \\
& = -g - \frac{\ddot{x}}{2ax} \\
4a^2 x (\dot{x}^2 + x\ddot{x}) & = -2gax - \ddot{x}
\end{align*}
Thus, the equations of motion for $x$ and $y$ are
\begin{align*}
0 & = (1 + 4a^2 x^2) \ddot{x} + 4a^2 x \dot{x}^2 + 2gax \\
y & = ax^2
\end{align*}

\item
The constraint is $f(x,y) = x^2 + y^2 = r^2$, so we have
\begin{align*}
\td{}{t} \pd{\Lg}{\dot{x}} - \pd{\Lg}{x} & = \lambda \pd{f}{x} \\
m \ddot{x} & = 2\lambda x \\
\td{}{t} \pd{\Lg}{\dot{y}} - \pd{\Lg}{y} & = \lambda \pd{f}{y} \\
m \ddot{y} + mg & = 2\lambda y \\
\end{align*}
This time, we use $\lambda' = 2\lambda/m$ to clean things up:
\begin{align*}
\ddot{x} & = \lambda' x \\
\ddot{y} + g & = \lambda' y
\end{align*}
We then eliminate $\lambda'$ to get
\begin{align*}
\frac{\ddot{x}}{x} & = \frac{\ddot{y} + g}{y}
\end{align*}
At this point we're technically done with part (b); the equations of motion are
\begin{align*}
\frac{\ddot{x}}{x} & = \frac{\ddot{y} + g}{y} \\
x^2 + y^2 & = r^2
\end{align*}


\item



\end{enumerate}


% In terms of both $x$ and $y$, the Lagrangian is
% \begin{align*}
% \Lg & = \frac{1}{2} m (\dot{x}^2 + \dot{y}^2) - mgy
% \end{align*}
% 
% \begin{enumerate}
% \item
% With $y = ax^2$, we have $\dot{y} = 2ax \dot{x}$, so the Lagrangian reads
% \begin{align*}
% \Lg & = \frac{1}{2} m (\dot{x}^2 + 4a^2 x^2 \dot{x}^2) - mgax^2 \\
% & = \frac{1}{2} m (1 + 4ax^2) \dot{x}^2 - mgax^2
% \end{align*}
% The equation of motion is then
% \begin{align*}
% \td{}{t} \pd{\Lg}{\dot{x}} & = \pd{\Lg}{x} \\
% \td{}{t} \left( m (1 + 4ax^2) \dot{x} \right) - 4max \dot{x}^2 & = - 2mgax
% \end{align*}
% 
% 
% \end{enumerate}
% 


\end{solution}



\newcommand{\oo}{\boldsymbol{\omega}}
\newcommand{\ee}{\mathbf{\hat{e}}}
\begin{solution}{2018.3}
Note: this problem was covered in the P521 lecture on September 21, 2020.
\begin{enumerate}
\item
We use
\begin{align*}
m \left( \td{^2 \rrr}{t^2} \right)_{body} & = \FF_e - m \left( \td{^2 \mathbf{a}}{t^2} \right)_{inertial}
- 2m \oo \times \left( \td{\rrr}{t} \right)_{body}
- m \oo \times (\oo \times \rrr) - m \td{\oo}{t} \times \rrr
\end{align*}
To first order in $\oo$, this is
\begin{align*}
m \ddot{\rrr} & = \FF_e - 2m \oo \times \dot{\rrr}
\end{align*}
In the given coordinates,
\begin{align*}
\oo & = \omega ( \cos \theta \ee_3 - \sin \theta \ee_1 ) \\
\FF_e & = -mg \ee_3
\end{align*}
We can then split the equation of motion out into its components:
\begin{align*}
\ddot{x} & = 2\dot{y} \omega \cos \theta \\
\ddot{y} & = -2\dot{x} \omega \cos \theta - 2\dot{z} \omega \sin \theta \\
\ddot{z} & = -g + 2 \dot{y} \omega \sin \theta
\end{align*}
Working order by order, we write
\begin{align*}
\rrr(t) & = \rrr_0(t) + \omega \tau \rrr_1(t) + \OO(\omega^2)
\end{align*}
where $\rrr_0$ and $\rrr_1$ are independent of $\omega$.
The order $\omega^0$ equations read
\begin{align*}
\ddot{x}_0 & = 0 \\
\ddot{y}_0 & = 0 \\
\ddot{z}_0 & = -g \\
\rrr_0(t) & = \left( v_0 t - \frac{g}{2} t^2 \right) \ee_3
\end{align*}
The time taken to reach the ground is then
\begin{align*}
\tau & = \frac{2v_0}{g}
\end{align*}
The order $\omega^1$ equations read
\begin{align*}
\omega \tau \ddot{x}_1 & = 2\dot{y}_0 \omega \cos \theta \\
\omega \tau \ddot{y}_1 & = -2\dot{x}_0 \omega \cos \theta - 2\dot{z}_0 \omega \sin \theta \\
\omega \tau \ddot{z}_1 & = 2\dot{y}_0 \omega \sin \theta
\end{align*}
Since $x_0(t) = y_0(t) = 0$, this simplifies to
\begin{align*}
\omega \tau \ddot{x}_1 & = 0 \\
\omega \tau \ddot{y}_1 & = -2(v_0 - gt) \omega \sin \theta \\
\omega \tau \ddot{z}_1 & = 0 \\
y_1(t) & = y_1(0) + \dot{y}_1(0) t - \frac{1}{\tau} \left( v_0 t^2 - \frac{g}{3} t^2 \right) \sin \theta
\end{align*}
Since $y(0) = 0$ and $\dot{y}(0) = 0$, the same must be true of $y_1$, giving us
\begin{align*}
y_1(t) & = - \frac{1}{\tau} \left( v_0 t^2 - \frac{g}{3} t^2 \right) \sin \theta
\end{align*}
Put all together, this gives us
\begin{align*}
\rrr(t) & = \left( v_0 t - \frac{g}{2} t^2 \right) \ee_3
- \left( v_0 t^2 - \frac{g}{3} t^2 \right) \omega \sin \theta \ee_2 + \OO(\omega^2)
\end{align*}
Therefore, the Coriolis deflection when the particle hits the ground is
\begin{align*}
\rrr(\tau) & = - \left( v_0 \tau^2 - \frac{g}{3} \tau^2 \right) \omega \sin \theta \ee_2 \\
& = -\frac{4}{3} \frac{v_0^3}{g^2} \omega \sin \theta \ee_2
\end{align*}

\item
If the particle is dropped from rest at height $h$ instead, we have
\begin{align*}
\rrr_0(t) & = \left( h - \frac{g}{2} t^2 \right) \ee_3 \\
\tau & = \sqrt{\frac{2h}{g}} \\
\omega \tau \ddot{y}_1 & = -2 (-gt) \omega \sin \theta \\
y_1(t) & = \frac{gt^3}{3\tau} \omega \sin \theta \\
\rrr(t) & = \left( h - \frac{g}{2} t^2 \right) \ee_3 + \frac{1}{3} gt^3 \omega \sin \theta + \OO(\omega^2)
\end{align*}
and the Coriolis deflection is
\begin{align*}
\rrr(\tau) & = \frac{1}{3} g \left( \frac{2h}{g} \right)^{3/2} \omega \sin \theta \ee_2
\end{align*}





\end{enumerate}

\end{solution}




\begin{solution}{2017.2}
Let $F(t)$ be the force applied to the sphere by the block via friction at time $t$.
Newton's laws tell us
\begin{align*}
m \ddot{x} & = F(t) \\
M \ddot{X} & = -F(t) + f(t)
\end{align*}
Additionally, by conservation of energy, the total kinetic energy of the system must be equal to the work done on the system.
The total kinetic energy is given by
\begin{align*}
T & = \frac{1}{2} M \dot{X}^2 + \frac{1}{2} m \dot{x}^2 + \frac{1}{2} I \omega^2 \\
& = \frac{1}{2} M \dot{X}^2 + \frac{1}{2} m \dot{x}^2 
+ \frac{1}{2} \left( \frac{2}{5} m R^2 \right) \left( \frac{\dot{x}}{R} \right)^2 \\
& = \frac{1}{2} M \dot{X}^2 + \frac{7}{10} m \dot{x}^2 
\end{align*}
while the work done on the system between time zero and time $t$ is
\begin{align*}
W & = \int \FF \cdot d\rrr = \int_0^t f(t') \dot{X}(t') \, dt'
\end{align*}
This gives us a system of equations for $x(t)$ and $X(t)$:
\begin{align*}
M \ddot{X} + m \ddot{x} & = f(t) \\
\frac{1}{2} M \dot{X}^2 + \frac{7}{10} m \dot{x}^2 & = \int_0^t f(t') \dot{X}(t') \, dt'
\end{align*}
From the first equation, and the fact that $\dot{x}(0) = \dot{X}(0) = 0$, we have
\begin{align*}
M\dot{X} + m \dot{x} & = - \frac{a}{\omega} \cos \omega t
\end{align*}
If we differentiate both sides of the second equation, we get
\begin{align*}
M \dot{X} \ddot{X} + \frac{7}{5} m \dot{x} \ddot{x} & = f(t) \dot{X}
\end{align*}
Substituting what we know about $\dot{X}$ and $\ddot{X}$, we have
\begin{align*}
\left( -m \dot{x} - \frac{a}{\omega} \cos \omega t \right) 
\frac{1}{M} (f(t) - m \ddot{x}) + \frac{7}{5} m \dot{x} \ddot{x}
& = -f(t) \frac{1}{M} \left( m \dot{x} + \frac{a}{\omega} \cos \omega t \right) \\
\left( m \dot{x} + \frac{a}{\omega} \cos \omega t \right) \frac{m}{M} \ddot{x}
+ \frac{7}{5} m \dot{x} \ddot{x} & = 0 \\
\left( \frac{m^2}{M} + \frac{7}{5} m \right) \dot{x} & = -\frac{a}{\omega} \cos \omega t \\
\left( \frac{m^2}{M} + \frac{7}{5} m \right) x & = -\frac{a}{\omega^2} \sin \omega t \\
\end{align*}

\end{solution}



\begin{solution}{2017.15}
In the rest frame of the pion, we have
\begin{align*}
\left( \begin{array}{c}
m_\pi \\
0 \\
\end{array} \right) & = \left( \begin{array}{c}
p_\nu' \\
p_\nu' \\
\end{array} \right) + \left( \begin{array}{c}
E_\mu \\
-p_\nu' \\
\end{array} \right) \\
E_\mu^2 & = m_\mu^2 + (p_\nu')^2 \\
& = (m_\pi - p_\nu)^2 \\
& = m_\pi^2 - 2m_\pi p_\nu' + (p_\nu')^2 \\
p_\nu' & = \frac{m_\pi^2 - m_\mu^2}{2m_\pi}
\end{align*}
In the lab frame, the neutrinos will clearly have the largest possible energy if they are emitted in the same direction as the momentum of the pion beam.
Therefore, in the lab frame, the four momentum of the neutrinos with maximum energy is
\begin{align*}
p_\nu^\alpha & = \left( \begin{array}{cccc}
\gamma & \gamma \beta & 0 & 0 \\
\gamma \beta & \gamma & 0 & 0 \\
0 & 0 & 1 & 0 \\
0 & 0 & 0 & 1 \\
\end{array} \right) \left( \begin{array}{c}
p_\nu' \\
p_\nu' \\
0 \\
0 \\
\end{array} \right) \\
& = \gamma (1 + \beta) \left( \begin{array}{c}
p_\nu' \\
p_\nu' \\
0 \\
0 \\
\end{array} \right)
\end{align*}
Finally, we must determine what $\gamma$ is for the pion beam.
Since 10 GeV $\gg m_\pi$, we may write
\begin{align*}
10 \mathrm{GeV} & \approx p_\pi = m_\pi \gamma \beta
\end{align*}
This gives us $\gamma \beta \approx 100$, which implies that $\beta \approx 1$.
Thus, the maximum energy that the neutrinos can have in the lab frame is
\begin{align*}
E_\nu & \approx (100 + 100) \frac{m_\pi^2 - m_\mu^2}{2m_\pi} \\
& \approx 6.0 \mathrm{GeV}
\end{align*}

\end{solution}


\begin{solution}{2016.1}
\begin{enumerate}
\item
The force $F$ is the force resulting from the potential
\begin{align*}
V & = \frac{1}{2} k r^2 = \frac{k}{2} (\rho^2 + z^2)
\end{align*}
Also, the kinetic energy of the particle can be written as
\begin{align*}
T & = \frac{m}{2} \left( \dot{z}^2 + R^2 \dot{\phi}^2 \right)
\end{align*}
Thus, the Lagrangian for this system is
\begin{align*}
\Lg & = T - V 
= \frac{m}{2} \left( \dot{z}^2 + R^2 \dot{\phi}^2 \right) - \frac{k}{2} (R^2 + z^2)
\end{align*}

\item
Lagrange's equations read
\begin{align*}
\td{}{t} \left( \pd{\Lg}{\dot{\phi}} \right) & = \pd{\Lg}{\phi} \\
0 & = \td{}{t} \left( mR^2 \dot{\phi} \right) \\
& = mR^2 \ddot{\phi} \\
\td{}{t} \left( \pd{\Lg}{\dot{z}} \right) & = \pd{\Lg}{z} \\
-kz & = \td{}{t} (m \dot{z}) = m\ddot{z}
\end{align*}

\item
Cleaning things up a bit, we have
\begin{align*}
0 & = mR^2 \ddot{\phi} \\
0 & = m\ddot{z} + kz
\end{align*}
Since $\td{}{t} (mR^2 \dot{\phi}) = 0$, we identify $mR^2 \dot{\phi}$ as a conserved quantity, and recognize that it is the angular momentum of the particle.
Defining $\omega^2 = k/m$, the $z$ equation gives us
\begin{align*}
z(t) & = A \cos \omega t + B \sin \omega t
\end{align*}
where $A$ and $B$ are determined by the initial conditions.
Thus, the particle exhibits harmonic oscillation around $z = 0$, and it revolves around the cylinder at a constant angular velocity.

\end{enumerate}

\end{solution}



\begin{solution}{2016.3}
\begin{enumerate}
\item
Let the equilibrium separation between the light atoms and the heavy atom be $L$.
The total kinetic energy is clearly
\begin{align*}
T & = \frac{1}{2} \left( M \dot{X}^2 + m \dot{x}_1^2 + m \dot{x}_2^2 \right)
\end{align*}
while the total potential energy is
\begin{align*}
V & = \frac{k}{2} \left( (X - x_1 - L)^2 + (x_2 - X - L)^2 \right)
\end{align*}
where I have assumed that $x_1 < X < x_2$.

\item
Let
\begin{align*}
x_1 & = x_1^0 + \eta_1 \\
x_2 & = x_2^0 + \eta_2 \\
X   & =   X^0 + \eta_3 \\
\end{align*}
where $x_1^0$, $x_2^0$, and $X^0$ are the equilibrium values of $x_1$, $x_2$, and $X$, and $\eta_i$ are their fluctuations from equilibrium.
We then re-write the kinetic and potential energy as
\begin{align*}
T & = \frac{M}{2} \dot{\eta}_3^2 + \frac{m}{2} (\dot{\eta}_1^2 + \dot{\eta}_2^2) \\
V & = \frac{k}{2} \left( (\eta_3 - \eta_1)^2 + (\eta_2 - \eta_3)^2 \right) \\
& = \frac{k}{2} \left( 2\eta_3^2 + \eta_1^2 + \eta_2^2 - 2\eta_1 \eta_3 - 2\eta_2 \eta_3 \right) \\
\end{align*}
Thus, the mass and potential matrices are
\begin{align*}
\underline{m} & = \left( \begin{array}{ccc}
m & 0 & 0 \\
0 & m & 0 \\
0 & 0 & M \\
\end{array} \right) \\
\underline{v} & = \left( \begin{array}{ccc}
k & 0 & -k \\
0 & k & -k \\
-k & -k & 2k \\
\end{array} \right)
\end{align*}
The normal mode frequencies are then found via
\begin{align*}
0 & = \det ( \underline{v} - \omega^2 \underline{m} ) \\
& = \det \left( \begin{array}{ccc}
k - \omega^2 m & 0 & -k \\
0 & k - \omega^2 m & -k \\
-k & -k & 2k - \omega^2 M \\
\end{array} \right) \\
& = (k - \omega^2 m) ((k - \omega^2 m) (2k - \omega^2 M) - k^2) - k(k(k - \omega^2 m)) \\
& = (k - \omega^2 m) \left[ (k - \omega^2 m) (2k - \omega^2 M) - k^2 - k^2 \right] \\
& = (k - \omega^2 m) \left[ \omega^4 m M - 2k \omega^2 m - k \omega^2 M \right] \\
& = \omega^2 (k - \omega^2 m) (m M \omega^2 - 2km - kM) \\
\omega^2 & = 0, \frac{k}{m}, \frac{2km + kM}{mM}
\end{align*}
The frequency $\omega = 0$ corresponds to the molecule moving with constant velocity and not oscillating.
The frequency $\omega^2 = k/m$ corresponds to the two bonds oscillating in phase (i.e. $x_2 - X = X - x_1$) because this case is exactly analogous to the particles of mass $m$ being attached to a fixed point by a spring with spring constant $k$.
Therefore, the last frequency must correspond to the atoms oscillating out of phase (i.e. $x_2 - X$ is maximized when $X - x_1$ is minimized).

\end{enumerate}

\end{solution}



\begin{solution}{2015.2}
\begin{enumerate}
\item
The energy of the particle at any time is 
\begin{align*}
E & = \frac{1}{2} m v^2 + \frac{\beta}{r^2}
\end{align*}
This is a conserved quantity, and it is equal to the initial energy $E_0 = \frac{1}{2} m v_0^2$.
Additionally, the angular momentum of the particle is
\begin{align*}
\ell & = mrv \sin \theta
\end{align*}
where $\theta$ is the angle between the radius vector and the velocity vector.
This is also a conserved quanitity, and it is equal to the initial angular momentum $mbv_0$.
Since $\rrr$ and $\vv$ are perpendicular when the particle makes its closest approach, we have a system equations for $R_{min}$, the distance of closest approach, and $v_{min}$, the speed of the particle at closest approach:
\begin{align*}
E & = \frac{1}{2} mv_{min}^2 + \frac{\beta}{R_{min}^2} \\
R_{min} v_{min} & = b v_0 \\
v_{min} & = \frac{bv_0}{R_{min}} \\
\frac{1}{2} mv_0^2 & = \frac{1}{2} m \left( \frac{bv_0}{R_{min}} \right)^2 + \frac{\beta}{R_{min}^2} \\
& = \frac{1}{2} m v_0^2 \frac{b^2}{R_{min}^2} + \frac{\beta}{R_{min}^2} \\
E & = E \frac{b^2}{R_{min}^2} + \frac{\beta}{R_{min}^2} \\
1 & = \frac{b^2 + \beta/E}{R_{min}^2} \\
R_{min} & = \sqrt{b^2 + \frac{\beta}{E}}
\end{align*}

\item
The speed of the particle when it makes its closest approach is then
\begin{align*}
v_{min} & = bv_0 \left( b^2 + \frac{\beta}{E} \right)^{-1/2}
\end{align*}



\end{enumerate}

\end{solution}



\begin{solution}{2015.15}
\begin{enumerate}
\item
This is not possible. It is clearly energetically forbidden in the rest frame of the electron.

\item
This is also not possible. Since the photon's energy is frame dependent, there exists a reference frame where the energy of the photon is less than the rest mass of an electron, making the process energetically forbidden.

\end{enumerate}

\end{solution}



\begin{solution}{2014.1}
\begin{enumerate}
\item
Let the horizontal position of $m_1$ be $x_1$, and let the angle between the pendulum arm and the vertical be $\phi$.
Denoting the $x$ and $y$ coordinates of $m_2$ as $x_2$ and $y_2$, we have
\begin{align*}
T & = \frac{1}{2} m_1 \dot{x}_1^2 + \frac{1}{2} m_2 (\dot{x}_2^2 + \dot{y}_2^2)
\end{align*}
We can write $x_2$ and $y_2$ in terms of $x_1$ and $\phi$:
\begin{align*}
x_2 & = x_1 + \ell \sin \phi \\
y_2 & = -\ell \cos \phi
\end{align*}
(taking the zero point for $y$ to be at the height of $m_1$), so
\begin{align*}
\dot{x}_2 & = \dot{x}_1 + \ell \dot{\phi} \cos \phi \\
\dot{y}_2 & = \ell \dot{\phi} \sin \phi
\end{align*}
Additionally, the potential energy of the system is
\begin{align*}
V & = m_2 g y_2 = -m_2 g \ell \cos \phi
\end{align*}
Thus, the Lagrangian for this system is
\begin{align*}
\Lg & = \frac{1}{2} m_1 \dot{x}_1^2 
+ \frac{1}{2} m_2 \left[ \left( \dot{x}_1 + \ell \dot{\phi} \cos \phi \right)^2 + (\ell \dot{\phi} \sin \phi)^2 \right]
+ m_2 g \ell \cos \phi
\end{align*}

\item
The equations of motion are then
\begin{align*}
\td{}{t} \left( \pd{\Lg}{\dot{x}_1} \right) & = \pd{\Lg}{x_1} \\
0 & = \td{}{t} \left[ m_1 \dot{x}_1 + m_2 \left( \dot{x}_1 + \ell \dot{\phi} \cos \phi \right) \right] \\
\td{}{t} \left( \pd{\Lg}{\dot{\phi}} \right) & = \pd{\Lg}{\phi} \\
\pd{\Lg}{\phi} & = 
m_2 \left[ \left( \dot{x}_1 + \ell \dot{\phi} \cos \phi \right) (-\ell \dot{\phi} \sin \phi)
+ (\ell \dot{\phi})^2 \sin \phi \cos \phi
\right] - m_2 g \ell \sin \phi \\
& = -m_2 \ell \dot{x}_1 \dot{\phi} \sin \phi - m_2 g\ell \sin \phi \\
\pd{\Lg}{\dot{\phi}} & = m_2 \left[ \left( \dot{x}_1 + \ell \dot{\phi} \cos \phi \right) \ell \cos \phi
+ \dot{\phi} \ell^2 \sin^2 \phi \right] \\
& = m_2 \left[ \dot{x}_1 \ell \cos \phi + \ell^2 \dot{\phi} \right] \\
\td{}{t} \left( \pd{\Lg}{\dot{\phi}} \right)
& = m_2 \left[ \ddot{x}_1 \ell \cos \phi - \dot{x}_1 \ell \dot{\phi} \sin \phi + \ell^2 \ddot{\phi} \right] \\
-g\ell \sin \phi & = \ddot{x}_1 \ell \cos \phi + \ell^2 \ddot{\phi}
\end{align*}

\item
TODO: normal modes

\end{enumerate}

\end{solution}




\begin{solution}{2014.3}
Assuming a central potential, we have
\begin{align*}
\phi & = \phi_0 \pm \frac{\ell}{\sqrt{2m}} \int dr \, r^{-2} [E - V_{eff}(r)]^{-1/2}
\end{align*}
where $V_{eff}(r) = V(r) + \frac{\ell^2}{2mr^2}$. Using $u = 1/r$, we have
\begin{align*}
du & = -\frac{1}{r^2} dr \\
\phi & = \phi_0 \pm \frac{\ell}{\sqrt{2m}} \int du \, \left( E - V(r=1/u) - \frac{\ell^2}{2m} u^2 \right)^{-1/2} \\
& = \phi_0 \pm \int du \, \left( \frac{2mE}{\ell^2} - \frac{2m}{\ell^2} V(u) - u^2 \right)^{-1/2}
\end{align*}
If the trajectory is a log spiral of the form $r = k e^{\alpha \phi}$, then
\begin{align*}
\phi & = \frac{1}{\alpha} \log \frac{r}{k}
\end{align*}
Therefore, we seek $V$ such that
\begin{align*}
\frac{1}{\alpha} \log \frac{r}{k} = -\frac{1}{\alpha} \log ku 
& = \pm \int du \, \left( \frac{2mE}{\ell^2} - \frac{2m}{\ell^2} V(u) - u^2 \right)^{-1/2}
\end{align*}
Differentiating both sides with respect to $u$ yields
\begin{align*}
-\frac{1}{\alpha u} & = \pm \left( \frac{2mE}{\ell^2} - \frac{2m}{\ell^2} V(u) - u^2 \right)^{-1/2} \\
\frac{1}{\alpha^2 u^2} & = \frac{2mE}{\ell^2} - \frac{2m}{\ell^2} V(u) - u^2 \\
\frac{2m}{\ell^2} V(u) & = \frac{2mE}{\ell^2} - \frac{1}{\alpha^2 u^2} - u^2 \\
V(r) & = E - \frac{\ell^2}{2m \alpha^2} r^2 - \frac{\ell^2}{2mr^2}
\end{align*}
Therefore, the necessary force law is
\begin{align*}
\FF & = -\del V = \frac{\ell^2}{m} \left( \frac{r}{\alpha^2} - \frac{1}{r^3} \right) \rr
\end{align*}

\end{solution}


\begin{solution}{2013.2}
Let the angle each pendulum makes with the vertical be $\phi_i$.
Taking $y = 0$ at the tops of the pendulums, we have
\begin{align*}
x_i & = C_i + L \sin \phi_i \\
y_i & = -L \cos \phi_i \\
\dot{x}_i & = L \dot{\phi}_i \cos \phi_i \\
\dot{y}_i & = L \dot{\phi}_i \sin \phi_i
\end{align*}
This lets us write the kinetic and potential energy as
\begin{align*}
T & = \frac{1}{2} \sum_{i=1}^3 m \left[ 
\left( L \dot{\phi}_i \cos \phi_i \right)^2 + \left( L \dot{\phi}_i \sin \phi_i \right)^2 \right] \\
V & = -\sum_{i=1}^3 mgL \cos \phi_i + 
\frac{1}{2} k \left[ (\sqrt{(x_2 - x_1)^2 + (y_2 - y_1)^2} - L_{12})^2 
+ (\sqrt{(x_3 - x_2)^2 + (y_3 - y_2)^2} - L_{23})^2 \right]
\end{align*}
where $L_{12}$ and $L_{13}$ are the equilibrium lengths of the springs connecting pendulums 1 and 2 and pendulums 2 and 3.

Writing $\phi_i = \phi_i^0 + \eta_i$, where $\phi_i^0$ are the equilibrium values of the $\phi_i$'s, we have to second order in the displacements from equilibrium
\begin{align*}
T & = \frac{1}{2} \sum_{i=1}^3 m \left[ 
\left( L \dot{\eta}_i \cos (\phi_i^0 + \eta_i) \right)^2 + \left( L \dot{\eta}_i \sin (\phi_i^0 + \eta_i) \right)^2 \right] \\
& \approx \frac{1}{2} \sum_{i=1}^3 m \left[ L^2 \dot{\eta}_i^2 (\cos^2 \phi_i^0 + \sin^2 \phi_i^0) \right] \\
& = \frac{1}{2} \sum_{i=1}^3 m L^2 \dot{\eta}_i^2 \\
V & = -\sum_{i=1}^3 mgL \cos (\phi_i^0 + \eta_i) \\
& + \frac{1}{2} k \left[ \left( \sqrt{(C_2 + L \sin (\phi_2^0 + \eta_2) 
- C_1 - L \sin (\phi_1^0 + \eta_1))^2 
+ (-L \cos (\phi_2^0 + \eta_2) + L \cos (\phi_1^0 + \eta_1))^2} - L_{12} \right)^2 \right. \\
& \left.
+ \left( \sqrt{(C_3 + L \sin (\phi_3^0 + \eta_3) 
- C_2 - L \sin (\phi_2^0 + \eta_2))^2 
+ (-L \cos (\phi_3^0 + \eta_3) + L \cos (\phi_2^0 + \eta_2))^2} - L_{23} \right)^2 \right]
% & \approx -\sum_{i=1}^3 mgL (\cos (\phi_i^0) (1-\eta_i^2/2) - \eta_i \sin \phi_i^0) \\
% & + \frac{1}{2} k \left[ \left( 
% \sqrt{(x_2^0 - L\sin \phi_2^0 \eta_2^2/2 + L \eta_2 \cos \phi_2^0
% - x_1^0 + L\sin \phi_1^0 \eta_1^2/2 - L \eta_1 \cos \phi_1^0)^2
% + 
\end{align*}
To proceed, note that the equilibrium values of $\phi_i$ cannot be determined from the information given in the problem.
Therefore, I assume that the normal mode frequencies do not depend on them, allowing me to choose $\phi_i^0 = 0$ for convenience.
To second order in the $\eta_i$'s, we then have
\begin{align*}
V & \approx -mgL \sum_{i=1}^3 \left( 1 + \frac{1}{2} \eta_i^2 \right) \\
& + \frac{k}{2} \left[ \left( \sqrt{ (C_2 + L \eta_2 - C_1 - L\eta_1)^2 
+ (-L + L\eta_2^2/2 + L - L\eta_1^2/2)^2} - L_{12} \right)^2 \right. \\
& + \left. \left( \sqrt{ (C_3 + L \eta_3 - C_2 - L\eta_2)^2 
+ (-L + L\eta_3^2/2 + L - L\eta_2^2/2)^2} - L_{23} \right)^2 \right]
\end{align*}
Note that using my assumption about $\phi_i^0$, we have $L_{12} = C_2 - C_1$ and $L_{23} = C_3 - C_2$.
Also, we only have to work to linear order inside the parenthesis to retain second order in the $\eta_i$'s overall.
Continuing, we have
\begin{align*}
V & \approx \mathrm{const} + \frac{1}{2} mgL \sum_{i=1}^3 \eta_i^2 \\
& + \frac{k}{2} \left[ \left( \sqrt{ L_{12}^2 + 2L_{12} L (\eta_2 - \eta_1)} - L_{12} \right)^2 \right. \\
& + \left. \left( \sqrt{ L_{23}^2 + 2L_{23} L (\eta_3 - \eta_2)} - L_{23} \right)^2 \right] \\
& \approx \mathrm{const} + \frac{1}{2} mgL \sum_{i=1}^3 \eta_i^2
+ \frac{k}{2} \left[ (L_{12} + L (\eta_2 - \eta_1) - L_{12})^2
+ (L_{23} + L (\eta_3 - \eta_2) - L_{23})^2 \right] \\
& \approx \mathrm{const} + \frac{1}{2} mgL \sum_{i=1}^3 \eta_i^2
+ \frac{1}{2} kL^2 \left[ \eta_2^2 - 2\eta_1 \eta_2 + \eta_1^2
+ \eta_3^2 - 2\eta_3 \eta_2 + \eta_2^2 \right]
\end{align*}
Thus,
\begin{align*}
\underline{m} & = \left( \begin{array}{ccc}
mL^2 & 0 & 0 \\
0 & mL^2 & 0 \\
0 & 0 & mL^2 \\
\end{array} \right) \\
\underline{v} & = \left( \begin{array}{ccc}
mgL + kL^2 & kL^2 & 0 \\
kL^2 & mgL + 2kL^2 & kL^2 \\
0 & kL^2 & mgL + kL^2 \\
\end{array} \right)
\end{align*}
and the normal mode frequencies are found with
\begin{align*}
0 & = \det [ \underline{v} - \omega^2 \underline{m} ] \\
& = \det \left( \begin{array}{ccc}
mgL + kL^2 - \omega^2 mL^2 & kL^2 & 0 \\
kL^2 & mgL + 2kL^2 - \omega^2 mL^2 & kL^2 \\
0 & kL^2 & mgL + kL^2 - \omega^2 mL^2 \\
\end{array} \right) \\
& = -(mL^2 \omega^2 - mgL) (mL^2 \omega^2 - 3kL^2 - mgL) (mL^2 \omega^2 - mgL - kL^2) \\
\omega^2 & = \frac{g}{L}, \, 3 \frac{k}{m} + \frac{g}{L}, \, \frac{k}{m} + \frac{g}{L}
\end{align*}

\end{solution}



\begin{solution}{2013.15}
Let $\beta_1$ refer to the velocity of the first particle, $\beta_2$ to the velocity of the first two particles after they merge, and $\beta_3$ to the velocity of the final particle.
Considering the four momenta before and after the first collision, we have
\begin{align*}
\left( \begin{array}{c}
\gamma_1 m \\
\gamma_1 \beta_1 m \\
\end{array} \right) + \left( \begin{array}{c}
m \\
0 \\
\end{array} \right) & = \left( \begin{array}{c}
\gamma_2 m_2 \\
\gamma_2 \beta_2 m_2 \\
\end{array} \right) \\
(\gamma_1 +1) m & = \gamma_2 m_2 \\
\gamma_1 \beta_1 m & = \gamma_2 \beta_2 m_2
\end{align*}
Considering the four momenta before and after the second collision, we have
\begin{align*}
\left( \begin{array}{c}
\gamma_2 m_2 \\
\gamma_2 \beta_2 m_2 \\
\end{array} \right) + \left( \begin{array}{c}
m \\
0 \\
\end{array} \right) & = \left( \begin{array}{c}
\gamma_3 M \\
\gamma_3 \beta_3 M \\
\end{array} \right) \\
(\gamma_1 + 2) m & = \gamma_3 M \\
\gamma_1 \beta_1 m & = \gamma_3 \beta_3 M \\
(\gamma_1 +2) m & = \gamma_3 \frac{\gamma_1 \beta_1 m}{\gamma_3 \beta_3} \\
\gamma_1 + 2 & = \gamma_1 \frac{\beta_1}{\beta_3} \\
\beta_3 & = \beta_1 \frac{\gamma_1}{\gamma_1 + 2} \\
\beta_3^2 & = \frac{\beta_1^2 \gamma_1^2}{(\gamma_1 + 2)^2} \\
& = \frac{\gamma_1^2 - 1}{(\gamma_1 + 2)^2} \\
\gamma_3 & = \left( 1 - \beta_3^2 \right)^{-1/2} \\
M & = m \frac{\gamma_1 + 2}{\gamma_3} \\
\end{align*}

\begin{enumerate}
\item
Using $\beta_1 = 0.8$, we have
\begin{align*}
\gamma_1 & = \frac{5}{3} \\
\beta_3 & = \frac{4}{11} \\
\gamma_3 & = \frac{11}{\sqrt{105}} \\
\frac{M}{m} & = \frac{\gamma_1 + 2}{\gamma_3} = \frac{\sqrt{105}}{3} \approx 3.4156
\end{align*}

\item
$V/c$ is just $\beta_3$, which is $4/11$.

\end{enumerate}


\end{solution}










\end{document}
